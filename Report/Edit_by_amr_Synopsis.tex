\RequirePackage{mmap}%
\documentclass[11pt]{article}
\usepackage[a4paper, hmargin={2.8cm, 2.8cm}, vmargin={2.5cm, 2.5cm}]{geometry}
\usepackage{eso-pic} % \AddToShipoutPicture
\usepackage{graphicx} % \includegraphicsud over
\usepackage[utf8]{inputenc} % æøå
\usepackage[T1]{fontenc} % mere æøå
\usepackage{verbatim} % så man kan skrive ren tekst
\usepackage{listings} % kode
\usepackage[round]{natbib} %citationer i chicago format
\usepackage{amsmath} % flere matematikkommandoer
\usepackage[danish]{babel} % orddeling
\usepackage[all]{xy} % den sidste (avancerede) formel i dokumentet
\usepackage{graphicx} %for billedhåndtering
\usepackage{listings} %algoritmer og programming language
\usepackage{fancyhdr} %header, footer
\usepackage{hyperref}
\usepackage{multirow}
\usepackage{pdfpages}
\usepackage{multicol}
\usepackage{fancyhdr}
\usepackage{tabularx}
\usepackage{tabularx}
\usepackage{longtable}
\newcommand{\citat}[2]{\begin{justify}\textit{``#1''}\hspace{0.1cm}\footnote{#2}\end{justify}}
\author{
 \textsc{\Large{Valdas Zabulionis}}\\
 \textsc{\Large{Rune Franch Pedersen}}\\
 \textsc{\Large{Amr El Sayed}}}
\title{
  \vspace{3cm}
  \textsc{\Huge{Bachelor Project}}\\
  \textsc{\Large{Online Algorithm Tutor: Web-Based Algorithm Visualization for Algorithm Learning}}\\
  \textsc{\Huge{\\Synopsis}}\\
 \textsc{\Large{2016 block 3-4}}
}
\begin{document}

%% Change `ku-farve` to `nat-farve` to use SCIENCE's old colors or
%% `natbio-farve` to use SCIENCE's new colors and logo.
\AddToShipoutPicture*{\put(0,0){\includegraphics*[viewport=0 0 700 600]{ku-farve}}}
\AddToShipoutPicture*{\put(0,602){\includegraphics*[viewport=0 600 700 1600]{ku-farve}}}

%% Change `ku-en` to `nat-en` to use the `Faculty of Science` header
\AddToShipoutPicture*{\put(0,0){\includegraphics*{kuen}}}

\clearpage\maketitle
\thispagestyle{empty}
\newpage
\begin{center}{\huge\textbf{Online Algorithm Tutor}}\newline \textit{\\DIKU - Bachelor Project}\end{center}
\hfill \break
\begin{tabular}{l l l }
\textbf{Studens Names} &: &Valdas Zabulionis - LHN100\\\\
&: &Rune Franch Pedersen - VQR730\\\\
&: &Amr Elsayed - VWJ159\\\\
\textbf{University} &:& University of Copenhagen\\\\
\textbf{Institution} &:& Department of Computer Science DIKU\\\\
\textbf{Supervisor} &:& Oleksandr Shturmov\\\\
\textbf{Period} &:& Block 3 - block 4\\\\
\textbf{Years} &:& 2016\\\\
\textbf{Pages} &:& \\
\end{tabular}
\\\\\\\\\\\\\\\\\\\\
\begin{center}{\huge\textbf{Certificate}}\end{center}

This is to certify that the work contained in the thesis entitled "Design and implementation interactive tools for teaching basic algorithm" by Valdas Zabulionis, Rune Franch Pedersen and Amr El sayed, has been carried out under our supervision and that this work has not been submitted elsewhere.\\\\\\\\\\\\
\begin{center}\noindent\rule{8cm}{0.4pt}%\end{center}

\begin{center}
\textbf{supervisor}\\
Oleksandr Shturmov \\
oleks@di.ku.dk \\
Department of Computer Science DIKU \\
University of Copenhagen
\end{center}

\newpage
\tableofcontents
\end{center}
\newpage
\section{Project Title}
%**  We are going to change ut soon **%
Online Algorithm Tutor: Web-Based Algorithm Visualization for Algorithm Learning   
\section{Problem definition}
Is it possible to create a program which helps students, who are relatively unfamiliar with algorithms and datastructures, to implement their own basic algorithms and visualize how they work.

\section{Project limitations}
We will only implement a few of the algorithms shown in the algorithm book, more could always be added at a later date.\\
We will not compute the running time of the students' pseudo-code algorithms, although it could be a part of the project. We have decided that we would leave the computation of running time to the students to find on their own, since it is an important part of algorithm design.\\
The program will not be able to execute algorithms, which were not built using building blocks that are defined in our program. So to import a pseudo-code algorithm, it will have to be written in a way that is recognized by our program.\\
No user manual.........\\
We will only perform internal testing of the program, and will not include any outside users or resources for testing of the functionality. \\
\section{Background}
begrundelse\\
four different courses, algorithms, hci, compilers, systemudvikling.\\
systemudvikling- life cycle development to implement a program\\
algorithms - pseudo-code of algorithms\\
compilers - how to read and compute user input/ lexer and context-free grammar\\
hci - help users learn by visualization and step by step computation with a user friendly interface design, error messages
\section{Method outline}
\section{Time schedule and work tasks}
arbejdsopgaver -\\
lexer(while, floor, +-*/), building blocks (afhængighed af lex, blocks), interface, step by step (interface, bblocks, lexer), pretty printing, correctness, list of basic algorithms?, metodiske overvejelser, relevant litteratur\\
\noindent\makebox[\linewidth]{\rule{\paperwidth}{0.4pt}}
\begin{center}\textbf{Work assignments}\end{center}
\noindent\makebox[\linewidth]{\rule{\paperwidth}{0.4pt}}

\textbf{Definition}: Building Blocks \\
\textbf{Product}: graphics and backend for the buttoms used for creating the algorithms aswell as implementing them in the interface.\\
\textbf{Resource Requirment}: open source langauge ( python, java script, HTML, CSS ) to implement the product, and browser to visualize the prodcut. \\
\textbf{Dependencies}: Interface - Parser\\
\textbf{Time Required}: 5-6 work days\\
\textbf{Dead line}: 10/ 04/ 2016\\
\noindent\makebox[\linewidth]{\rule{\paperwidth}{0.4pt}}

\textbf{Definition}: Interface\\
\textbf{Product}: graphics and backend for the application interface.\\
\textbf{Resource Requirment}: open source langauge ( python, java script, HTML, CSS ) to implement the product, and browser to visualize the prodcut. \\
\textbf{Dependencies}: NONE!\\
\textbf{Time Required}: 2-3 work days.\\
\textbf{Dead line}: 17 / 03/ 2016\\
\noindent\makebox[\linewidth]{\rule{\paperwidth}{0.4pt}}

\textbf{Definition}:  Parser \\
\textbf{product} : definiaion of the building blocks syntax\\
\textbf{Resource Requirment}: Introduction to Compiler Design - Torben Ægidius Mogensen\\
\textbf{Dependencies}: None\\
\textbf{Time Required}: 1-2 work days\\
\textbf{Dead line}: 17/ 03/ 2016\\
\noindent\makebox[\linewidth]{\rule{\paperwidth}{0.4pt}}

\textbf{Definition}:  Step by Step Simulation\\
\textbf{product}: graphics and backend for the step by step  algorithms simulation, aswell as implementing them in the interface.\\
\textbf{Resource Requirment}: open source langauge ( python, java script, HTML, CSS ) to implement the product, and browser to visualize the prodcut. \\
\textbf{Dependencies}: Interface\\
\textbf{Time Required}: 5-6 work days\\
\textbf{Dead line}: 15/ 05/ 2016\\
\noindent\makebox[\linewidth]{\rule{\paperwidth}{0.4pt}}

\textbf{Definition}:  Pretty Print\\
\textbf{product}: graphics and backend for printing the variables results and the return results of the functions, aswell as implementing them in the interface.\\
\textbf{resource requirment}: open source langauge ( python, java script, HTML, CSS ) to implement the product, and browser to visualize the prodcut. \\
\textbf{Dependencies}: Interface\\
\textbf{Time Required}: 8-10 work days\\
\textbf{Dead line}: 01/ 06/ 2016\\

\section{References}
oversætter bogen, algoritme bogen,\\
\url{http://pgbovine.net/publications/Online-Python-Tutor-web-based-program-visualization\_SIGCSE-2013.pdf}


\newpage
\section{Appandix A}
\end{document}
